\documentclass[12pt]{article}
\usepackage[margin=3cm]{geometry}
\usepackage{blindtext}
\usepackage{chngcntr}
\counterwithin*{section}{part}
\usepackage{enumitem}
\usepackage{listings}
\usepackage{float}
\usepackage{graphicx}
\usepackage{tikz}
\usepackage{subfig}
\usepackage{hyperref}
\usepackage{amsmath}
\usepackage{multicol}
\usepackage{tcolorbox}
\usepackage{xcolor}
\lstset{language=C++,
    basicstyle=\ttfamily,
    keywordstyle=\color{blue},
    stringstyle=\color{red},
    commentstyle=\color{green},
    morecomment=[l][\color{magenta}]{\#}
}

\hypersetup{
    colorlinks=true,
    linkcolor=blue,
    citecolor=green,
    urlcolor=red
}
\usetikzlibrary{quotes, angles, decorations.markings, intersections}
\usetikzlibrary{calc,patterns,angles,quotes, 3d, intersections, positioning, shapes, automata, positioning}
\newcommand{\tbox}[1]{\noindent\fbox{\parbox{\textwidth}{#1}}}
\title{OS CS219 Notes}
\author{}
\date{\today}
\begin{document}
\maketitle
\setlength{\parskip}{6pt}
\setlength{\parindent}{0pt}

\noindent\tbox{
    \begin{center}
    \textbf{\Huge Lecture 1}
    \end{center}
}


\noindent\tbox{
    \begin{center}
    \textbf{\Huge Lecture 2}
    \end{center}
}


\section{IP address}

Each connected device has an unique identifier to describe it.
This identifier is known as the IP address of a device. 

The IP address has a hierarchical structure. An example of an IP address is `72.85.5.25'. This can be thought of 
as being similar to a postal address where the country of your address is the highest level at which location is specified. 
After this the state, city, area narrow down your location more and the message travels in an organised way from one `level' to another.


\section{Message}


A message can be descibed in various granularities.
\begin{itemize}
    \item \textbf{Level 1:} Frame
    \item \textbf{Level 2:} Symbol
    \item \textbf{Level 3:} Packet
    \item \textbf{Level 4:} TCP $\rightarrow$ segment, UDP $\rightarrow$ Datagram
\end{itemize}


As we saw in the last Lecture a router has some number of `in' connections and `out' connections which are connected together depending on which 
`out' connection the message is supposed to be sent. This  `routing' is done by putting each incoming packets on queues corresponding to the outgoing connection we 
are supposed to send to.

This rerouting is done so that if the incoming rate to an out connection is greater than its outgoing rate we have the queue as a buffer.


Why do we just not make the queue very large to prevent these 'drop offs'
\begin{itemize}
    \item \textbf{Cost:} Memory is not Free so we need to be aware of the trade offs of increasing the queue size
    \item \textbf{End-End delay:} If the router has a buffer of say size $Q_{max}$ and an output rate of c $bits/sec$.
    Now if the queue is almost full and we get a new packet put in at the very end, the packet takes $\frac{Q_{max}}{c}$ time to be put on the next connection. 
    This is called the End-End delay and clearly a large queue increasing the worst case End-End delay
\end{itemize}

The total delay for a transmission of a packet through some K routers would be 
\[ Delay = \sum_{k = 1}^{k = K}\frac{Q_{max}^k}{c^k} + S_d + T_d\]

\(S_d\) is called the speed of light delay and \(T_d\) is the transmission delay. 
In most applications End-End delay is the significant bottle neck for the whole delay. Infact in some applications 
we prefer dropping packets inorder to not have high delays 

Most commercial networks have more routers than needed to have good back-ups to prevent the shutdown of their networks even if a router 
fails. How is this rerouting done?

\begin{itemize}[itemsep = 0cm, parsep = 0cm, topsep = 0cm]
    \item Readjusting weights:
    \item \textbf{Longest Prefix Match:} There is a router table present which gives us the ID corresponding to a router. When a packet has to make the choice between routers it takes the path with the longest prefix match. 
\end{itemize}


\section{Tansmission Control Protocol(Layer - 4 ):}


\section{Layers of a network}

\begin{itemize}[itemsep = 0cm, parsep = 0cm, topsep = 0cm]
    \item \textbf{L5:} Applications:Web (HTTP),Email (SMTP),Voice Calls (VOIP),Text messaging
    \item \textbf{L4:} TCP, UDP
    \item \textbf{L3:} IP (Internet Protocol)
    \item \textbf{L2:} WiFi, 4G, Ethernet
    \item \textbf{L1:} Wireless, Optic Fibre
\end{itemize}

\section{Layering and Design Protocols}
Any subproblem is handled by some protocol.

We have divided networks into 
5 layers. Specifically Application layer, Transmission Layer, Network Layer,Data Link Layer ,Physical Layer.  
This abstraction enables users to interact only with the layer they are concerned with in that layer without having to deal with the 
network as a whole. 

Some advantages of layering networks are:
\begin{itemize}[itemsep = 0cm, parsep = 0cm, topsep = 0cm]
    \item \textbf{Ease of development:} Only certain problems need to be dealt with at each layer
    \item \textbf{Debugging:} Ease in fixing new problems in each layer independently
    \item \textbf{Flexibility of Physical technologies, Applications:} As an example whatsapp as an application 
    only deals with that layer of the network. It doesn't interfere/have to deal with the particular intricacies of the 
    physical technology used by their users to connect to the network.
    \item \textbf{Ease of Modification:} We need to change only a particular layer to address problems associated with it 
\end{itemize}
\end{document}